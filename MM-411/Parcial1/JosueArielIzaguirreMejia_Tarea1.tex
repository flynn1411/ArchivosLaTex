\documentclass{article}
\usepackage[utf8]{inputenc}
\usepackage[T1]{fontenc}
\usepackage[margin=1in, headsep=0pt]{geometry}
\usepackage{amsmath}
\usepackage{mathtools}
\newcommand{\newLine}[3]{ #1 & = #2 && \text{(#3)}\\ }
\newcommand{\finalAnswer}[2]{ \Aboxed{ #1 & = #2 } }
\newcommand{\p}[1]{ \left( #1 \right) }
\newcommand{\brackets}[1]{ \left[ #1 \right] }
\newcommand{\newProblem}[2]{ \hspace{10mm} #1. $
\boldsymbol{ #2 }
$  \break }
\newcommand{\e}[1]{e^{#1}}
\newcommand{\homogenea}{
    u & = \frac{y}{x} \\
    y & = ux \\
    dy & = udx + xdu\\
}
\newcommand{\notEqual}[3]{ #1 & \neq #2 && \text{(#3)}\\ }
\newcommand{\pD}[2]{ \frac{\partial #1}{\partial #2} }
\allowdisplaybreaks
%\usepackage{showframe}
\geometry{letterpaper}
\title{Tarea Unidad 1 MM-411}
\author{Josué Ariel Izaguirre (20171034157)}
\date{Febrero 2020}
\begin{document}
    \maketitle
    \begin{flushleft}
        \textbf{1.Resuelva las siguientes ecuaciones diferenciales:} \break
        
        
        \hspace{10mm} a. $
            \boldsymbol{\left( 1+x^2 \right) \frac{dy}{dx} = \left(1+y \right)^2}
            $  \break

                \begin{align*}
                    ( 1+x^2 ) \frac{dy}{dx} & = (1+y )^2  &&\text{(1)}\\
                    [ ( 1+x^2 ) \frac{dy}{dx} & = (1+y)^2 ] {dx}{(1+x^2)(1+y)^2} &&\text{(2)}\\
                    \frac{dy}{(1+y)^2} & = \frac{dx}{1+x^2} &&\text{(3)}\\
                    \int_{}^{} \frac{dy}{(1+y)^2} & = \int_{}^{} \frac{dx}{1+x^2} &&\text{(4)}\\
                    -\frac{1}{1+y} + c_1 & = \tan ^ {-1} {x} + c_2 &&\text{(5)}\\
                    -\frac{1}{1+y} & = \tan ^ {-1} {x} + c_3, \quad (c_3 = c_2 - c_1) &&\text{(6)} \\
                    -1 - \tan ^ {-1} {x} - c_3 & = y ( \tan ^ {-1} {x} + c_{3} ) &&\text{(7)} \\
                    \Aboxed{y & = -1 - \frac{1}{tan ^{-1}x + c}}
                \end{align*}\\

        \hspace{10mm} b. $
            \boldsymbol{\frac{dy}{dx} = 6e^{2x-y} , y(0) = 0}
            $  \break

            \begin{align*}
                [ \frac{dy}{dx} & = 6\frac{e^{2x}}{e^y} ] (e^y dx) && \text{(1)} \\
                e^y dy & = 6e^{2x} dx && \text{(2)} \\
                \int e^y dy & = \int 6e^{2x} dx && \text{(3)} \\
                e^y & = 3e^{2x} + c && \text{(4)} \\
                y & = Ln(3e^{2x} + c) && \text{(5)} \\
                \\
                y(0) & = 0 && \text{(6.1)} \\
                0 & = Ln(3e^{2(0)} + c) && \text{(6.2)} \\
                1 & = 3 + c && \text{(6.3)} \\
                c & = -2 && \text{(6.4)}\\
                \\
                \Aboxed{y &= Ln(3e^{2x}-2)}
            \end{align*}

        \hspace{10mm} c. $
            \boldsymbol{ xy' + y = 3xy , y(1) = 0}
            $  \break

            \begin{align*}
                xy' & = 3xy - y  && \text{(1)}\\
                y' & = 3y - \frac{y}{x}  && \text{(2)}\\
                \frac{dy}{dx} & = 3y - \frac{y}{x}  && \text{(3)}\\
                [ \frac{dy}{dx} & = 3y - \frac{y}{x} ] (\frac{dx}{y})  && \text{(4)}\\
                \frac{dy}{y} & = (3 - \frac{1}{x})dx  && \text{(5)}\\
                \int \frac{dy}{y} & = \int (3 - \frac{1}{x})dx  && \text{(6)}\\
                ln(y) & = 3x - ln(x) + c  && \text{(7)}\\
                \Aboxed{y & = \frac{e^{3x + c}}{x}}\\
            \end{align*}

        \hspace{10mm} d. $
            \boldsymbol{ (1+x)y' + y = cosx , y(0) = 1}
            $  \break

            \begin{align*}
                [ (1+x)y' + y & = cosx ] (\frac{1}{1+x}) && \text{(1)}\\
                y' + \frac{y}{1+x} & = \frac{cosx}{1+x} && \text{(2)}\\
                \\
                \mu(x) & = e^{\int \frac{1}{1+x} dx} && \text{(3.1)}\\
                \mu(x) & = e^{ln(1+x)} && \text{(3.2)}\\
                \mu(x) & = 1+x && \text{(3.3)}\\
                \\
                [ y' + \frac{y}{1+x} & = \frac{cosx}{1+x} ] (1+x) && \text{(4)}\\
                (1+x)y' + y & = cosx && \text{(5)}\\
                \int \frac{d}{dx} [ (1+x)y ] & = \int cosx dx && \text{(6)}\\
                \newLine{(1+x)y}{senx + c}{7}
                \newLine{y}{\frac{senx + c}{1+x}}{8}
                \\
                \newLine{y(0)}{1}{9.1}
                \newLine{1}{ \frac{sen(0) + c}{1+(0)} }{9.2}
                \newLine{1}{ \frac{c}{1} }{9.3}
                \newLine{c}{1}{9.4}
                \\
                \finalAnswer{y}{\frac{senx + 1}{1+x}}
            \end{align*}

        \hspace{10mm} e. $
            \boldsymbol{ xyy' = y^2 + x\sqrt{4x^2 + y^2} }
            $  \break

            \begin{align*}
                \newLine{[ xy\frac{dy}{dx}} {y^2 x\sqrt{ 4x^2 + y^2} ] (dx) } {1}
                \newLine{xydy} {(y^2 x\sqrt{ 4x^2 + y^2})dx} {2}
                \newLine{xydy - (y^2 x\sqrt{ 4x^2 + y^2})dx } {0} {3}
                \\
                \newLine{u}{\frac{y}{x}}{4.1}
                \newLine{y}{ux}{4.2}
                \newLine{dy}{udx+xdu}{4.3}
                \\
                \newLine{ux^2 (udx + xdu) - (u^2 x^2 + x \sqrt{4x^2 + u^2 x^2})dx}{0}{5}
                \newLine{ x^2(u^2)dx + x^2(-u^2 - \sqrt{4-u^2})dx + x^3du }{0}{6}
                \newLine{ x^2(-\sqrt{4-u^2})dx }{-x^3du}{7}
                \newLine{ x^2(-\sqrt{4-u^2})dx }{-x^3du ] (\frac{1}{(-x^3)(-\sqrt{4-u^2})})}{8}
                \newLine{ \frac{dx}{x} }{ \frac{du}{ \sqrt{4-u^2}} }{9}
                \newLine{\int \frac{dx}{x} }{ \int \frac{du}{ \sqrt{4-u^2}} }{10}
                \newLine{ Ln(x) }{ \int \frac{du}{ \sqrt{4-u^2}} }{11.1}
                \\
                \newLine{2tan\theta = u}{ tan\theta = \frac{u}{2} }{11.1.a}
                \newLine{\sqrt{ 4tan^{2}\theta +4 }}{2sec\theta}{11.1.b}
                \newLine{2sec^{2}\theta d\theta}{du}{11.1.c}
                \newLine{ \int \frac{du}{sqrt{4-u^2}} } { \int \frac{ 2sec^{2}\theta }{ 2sec\theta } } {11.1.d}
                \newLine{ \int \frac{ 2sec^{2}\theta }{ 2sec\theta } } { \int sec\theta d\theta } {11.1.e}
                \newLine{ \int sec\theta d\theta } { ln(tan\theta + sec\theta) + c_1 } {11.1.f}
                \newLine{ \int sec\theta d\theta } { ln( \frac{u}{2} + \frac{ \sqrt{4-u^2} }{2} ) + c_1 } {11.1.g}
                \\
                \newLine{ Ln(x) } { ln( \frac{u + \sqrt{4-u^2} }{2} ) + c_1 } { 12 }
                \newLine{ x } { c \left( u + \sqrt{4-u^2}  \right), c=\frac{e^{c_1}}{2} } { 13 }
                \newLine{ x } { c \left( (\frac{y}{x}) + \sqrt{4-\frac{y^2}{x^2}} \right)  } { 14 }
                \newLine{ x } { c \p{ \frac{y}{x} + \frac{ \sqrt{4x^2 + y^2} }{x} } } { 15 }
                \newLine{ x } { c \p{ \frac{y + \sqrt{4x^2 + y^2} }{x} } } { 16 }
                \finalAnswer{ \frac{ x^2 }{ y + \sqrt{4x^2 + y^2} } }{c}
            \end{align*}

        \newProblem{f}{ x^2 y' = xy + x^2 e^{\frac{y}{x}} }
            \begin{align*}
                \newLine{ [ x^2 y' } { xy + x^2 e^{ \frac{y}{x} } ] \p{ \frac{1}{x^2} } } { 1 }
                \newLine{ \frac{dy}{dx} } { \frac{y}{x} + e^{ \frac{y}{x} } } { 2 }
                \newLine{ dy } { ( \frac{y}{x} + \e{\frac{y}{x}})dx } { 3 }
                \\
                \homogenea
                \\
                \newLine{ udx + xdu } { ( u + \e{u} )dx } { 4 }
                \newLine{ udx - ( u + \e{u} )dx } { xdu } { 5 }
                \newLine{ \e{u} dx } { xdu } { 6 }
                \newLine{ \int \frac{dx}{x} } { \int \e{-u} } { 7 }
                \newLine{ ln(x) } { -\e{-u} + c } { 8 }
                \newLine{ ln(x) } { -\e{-\frac{y}{x}} + c} { 9 }
                \finalAnswer{ ln(x) + \e{-\frac{y}{x}} } { c }
            \end{align*}

        \newProblem{ g }{ y' = \sqrt{y' = x + y +1 } }
            \begin{align*}
                \newLine{ dy } { \sqrt{ x + y + 1 }dx } { 1 }
                \\
                \newLine{ w } { x+y+1 } { 2.1 }
                \newLine{ dw } { dy } { 2.2 }
                \\
                \newLine{ dw } { \sqrt{w}dx } { 3 }
                \newLine{ \int \frac{dw}{\sqrt{w}} } {\int dx } { 4 }
                \newLine{ \frac{1}{2} \sqrt{w} } { x + c } { 5 }
                \newLine{ \frac{1}{2} \sqrt{x+y+1} } { x + c } { 6 }
                \newLine{ x + y+ 1 } { (x + c) ^ 2 } { 7 }
                \finalAnswer{ y }{ (x + c)^2 - x - 1}
            \end{align*}

        \newProblem{ h } { y' = y + y^3 }
            \begin{align*}
                \newLine{ y' - y } { y^3 } { 1 }
                \\
                \newLine{ u }{ y^{ 1 - 3 } }{ 2.1 }
                \newLine{ y } { u^{-\frac{1}{2}} } { 2.2 }
                \newLine{ y' } { -\frac{1}{2} u^{-\frac{3}{2}} u' } { 2.3 }
                \\
                \newLine{ -\frac{1}{2} u^{-\frac{3}{2}} u' - u^{-\frac{1}{2}} } { u^{-\frac{3}{2}} } { 3 }
                \newLine{ u' + 2u } { -1 } { 4 }
                \\
                \newLine{ \mu(x) } { \e{ \int 2 dx} } { 5.1 }
                \newLine{ \mu(x) } { \e{ 2x } } { 5.2 }
                \\
                \newLine{ [ u' +2u } { -1 ] \e{2x} } { 6 }
                \newLine{ u'\e{ 2x } + 2u\e{ 2x } } { -\e{2x} } { 7 }
                \newLine{ \frac{d}{dx} [ u\e{2x} ] } { -\e{2x} } { 8 }
                \newLine{ u\e{2x} } { \int -\e{2x} dx } { 9 }
                \newLine{ u\e{2x} } { -\frac{1}{2} \e{2x} + c } { 10 }
                \newLine{ u } { c\e{-2x} -\frac{1}{2} } { 11 }
                \newLine{ \frac{y}{x} } { c\e{-2x} -\frac{1}{2} } { 12 }
                \finalAnswer{ y } { \frac{ c }{ \e{2x}x } - \frac{ 1 }{ 2x }}
            \end{align*}

        \newProblem{ i }{ xy' + 6y = 3xy^{\frac{4}{3}} }
            \begin{align*}
                \newLine{ y' + 6\frac{y}{x} } { 3y^{ \frac{4}{3} } } { 1 }
                \\
                \newLine{ u } { y^{-\frac{1}{3}} } { 2.1 }
                \newLine{ y } { u^{ -3 } } { 2.2 }
                \newLine{ y' } { -3u^{-4} u' } { 2.3 }
                \\
                \newLine{ -3u^{-4}u' +6u^{-3}x^{-1} } { 3u^{-4} } { 3 }
                \newLine{ u' + 2ux^{-1} } { -1 } { 4 }
                \\
                \newLine{ \mu(x) } { \e{ 2\int \frac{dx}{x} }  } { 5.1 }
                \newLine{ \mu(x) } { 2\e{ln(x)} } { 5.2 }
                \newLine{ \mu(x) } { x^2 } { 5.3 }
                \\
                \newLine{ u'x^2 + 2xu } { -x^2 } { 6 }
                \newLine{ \frac{d}{dx}[ ux^2 ] }{ -x^2 }{7}
                \newLine{ ux^2 } { \int -x^2 dx } { 8 }
                \newLine{ ux^2 } { -\frac{1}{3}x^3 + c } { 9 }
                \newLine{ yx } { -\frac{1}{3}x^3 + c} { 10 }
                \finalAnswer{ y } { -\frac{1}{3}x^2 + \frac{c}{x} }
            \end{align*}
        
        \newProblem{j}{ \frac{dy}{dx} = \frac{2y-x+7}{4x-y-18} }
            \begin{align*}
                \newLine{ ( 4x-y-18 )dy } { (2y-x+7)dx } { 1 }
                \\
                    \begin{cases}
                        4x - y = 18\\
                        2y - x = -7
                    \end{cases}\\
                x = 3 && y = -2 \\
                x = u + 3 && y = v -2 \\
                dx = du && dy = dv\\
                \\
                \newLine{ ( 4u - 3v )dv } { (2v-u)du } {2}
                \newLine{ (2v-u)du - ( 4u - 3v )dv } { 0 } { 3 }
                \\
                \newLine{ w } { \frac{v}{u} } { 4.1 }
                \newLine{ v } { wu } { 4.2 }
                \newLine{ dv } { wdu + udw } { 4.3 }
                \\
                \newLine{ (2wu-u)du -(4u-3wu)(wdu+udw) } { 0 } {5}
                \newLine{ u(3w^2 -2w-1)du } { u^2(4-3w)dw } { 6 }
                \newLine{ \int \frac{du}{u} } { \int \frac{4-3w}{3w^2-2w-1} dw } { 7 }
                \\
                    \newLine{ \int \frac{4-3w}{3w^2-2w-1} dw  } { \frac{1}{4}\int \frac{1}{w-1}dw - \frac{15}{4}\int \frac{1}{3w+1} dw } { 8.1 }
                    \newLine{ \int \frac{4-3w}{3w^2-2w-1} dw } { \frac{1}{4}ln(w-1) - \frac{3}{4}ln(3w+1) } { 8.2 }
                \\
                \newLine{ ln(u) } { \frac{1}{4}ln(w-1) - \frac{3}{4}ln(3w+1) + c} { 9 }
                \newLine{ u } { C\p{w-1}^{\frac{1}{4}} \p{3w+1}^{-\frac{3}{4}} } { 10 }
                \newLine{ x-3 } { C \p{\frac{y+2}{x-3}-1}^{\frac{1}{4}} \p{3\p{\frac{y+2}{x-3}}+1}^{-\frac{3}{4}} } { 11 }
                \finalAnswer{ \p{x-3} \p{\frac{y+2}{x-3}-1}^{-\frac{1}{4}} \p{3\p{\frac{y+2}{x-3}}+1}^{\frac{3}{4}} }{ C }
            \end{align*}

        \newProblem{k}{ \frac{dy}{dx} = \frac{ x-y-1 }{ x+y+3 } }
        \begin{align*}
            \newLine{ ( x+y+3 )dy } { (x-y-1)dx } { 1 }
            \\
                \begin{cases}
                    x+y = -3\\
                    x-y = 1
                \end{cases}\\
            x = -1 && y = -2 \\
            x = u - 1 && y = v -2 \\
            dx = du && dy = dv\\
            \\
            \newLine{ (v+u)dv } { (u-v)du } {2}
            \newLine{ (u-v)du - ( v+u )dv } { 0 } { 3 }
            \\
            \newLine{ w } { \frac{v}{u} } { 4.1 }
            \newLine{ v } { wu } { 4.2 }
            \newLine{ dv } { wdu + udw } { 4.3 }
            \\
            \newLine{ (wu-u)du - (wu+u)(wdu+udw) } { 0 } {5}
            \newLine{ u(1-2w-w^2)du } { u^2(w+1)dw } { 6 }
            \newLine{ \int \frac{du}{u} } { \int \frac{w+1}{1-2w-w^2} dw } { 7 }
            \\
                \\
                \newLine{t}{1-2w-w^2}{8.1}
                \newLine{dt}{-2(1+w)dw}{8.2}
                \newLine{\frac{dt}{-2(1+w)}}{dw}{8.3}
                \\
                \newLine{ \int \frac{w+1}{1-2w-w^2} dw } { -\frac{1}{2} \int \frac{1}{t} dt } { 8.4 }
                \newLine{ -\frac{1}{2} \int \frac{1}{t} dt }{ -\frac{1}{2} ln(t) + c}{ 8.5 }
                \newLine{ \int \frac{w+1}{1-2w-w^2} dw } { -\frac{1}{2} ln(1-2w-w^2) + c } { 8.6 }
            \\
            \newLine{ ln(u) } { -\frac{1}{2} ln(1-2w-w^2) + c} { 9 }
            \newLine{ u } { C\p{1-2w-w^2}^{-\frac{1}{2}} } { 10 }
            \newLine{ x+1 } { C\p{1-2\p{\frac{y+2}{x+1}}-\p{\frac{y+2}{x+1}}}^{-\frac{1}{2}} } { 11 }
            \finalAnswer{ \p{x+1} \p{1-2\p{\frac{y+2}{x+1}}-\p{\frac{y+2}{x+1}}}^{\frac{1}{2}} }{ C }
        \end{align*}

        \newProblem{l}{ ( \e{x}seny + tany )dx + ( \e{x}cosy + xsec^{2}y )dy = 0 }
         \begin{align*}
             \newLine{ \frac{\partial M}{\partial y} = \e{x}cosy + sec^{2}y } { \frac{\partial N}{\partial x} = \e{x}cosy + sec^{2}y } { 1 }
             \\
             \newLine{F(x,y)}{\int \p{\e{x}seny + tany} dx + g(y)}{2.1}
             \newLine{F(x,y)}{ \e{x}seny + xtany + g(y) }{ 2.2 }
             \\
             \newLine{ \frac{\partial F}{\partial y} } { N(x,y) } { 3.1 }
             \newLine{ \e{x}cosy + xsec^{2}y + g'(y) } { \e{x}cosy + xsec^{2}y } { 3.2 }
             \newLine{ g'(y) } { 0 } { 3.3 }
             \newLine{ g(y) } { \int 0 dy } { 3.4 }
             \newLine{ g(y) } { C } { 3.5 }
             \\
             \newLine{ F(x,y) }{ \e{x}seny + xtany + C }{ 4 }
             \finalAnswer{ \e{x}seny + xtany }{ C }
         \end{align*}

         \newProblem{m}{ \p{x + tan^{-1}y}dx + \p{ \frac{x+y}{1+y^2} }dy = 0 }
            \begin{align*}
                \newLine{ \pD{M}{y} = \frac{1}{1+y^2} } { \pD{N}{x} = \frac{1}{1+y^2} } { 1 }
                \\
                    \newLine{ F(x,y) } { \int \p{x + tan^{-1}y}dx + g(y) } { 2.1 }
                    \newLine{ F(x,y) } { \frac{1}{2}x^2 + xtan^{-1}y + g(y) } { 2.2 }
                \\
                    \newLine{ \pD{F}{y} } { N(x,y) } { 3.1 }
                    \newLine{ \frac{x}{1+y^2} + g'(y) } { \frac{x}{1+y^2} + \frac{y}{1+y^2} } { 3.2 }
                    \newLine{ g'(y) } { \frac{y}{1+y^2} } { 3.3 }
                    \newLine{ g(y) } { \int \frac{y}{1+y^2}dy } { 3.3 }
                    \newLine{ g(y) } { \frac{1}{2}ln\p{1+y^2} } { 3.4 }
                \\
                \newLine{ F(x,y) } { \frac{1}{2}x^2 + xtan^{-1}y + \frac{1}{2}ln\p{1+y^2} } {4}
                \finalAnswer{ \frac{1}{2}x^2 + xtan^{-1}y + \frac{1}{2}ln\p{1+y^2} }{ C }
            \end{align*}
        
            \newProblem{ n }{ 2xydx + \p{y^2 - x^2}dy = 0}
                \begin{align*}
                    \homogenea
                    \\
                    \newLine{ (2x^2 u)dx + (x^2y^2 - x^2)\p{udx+xdu} } { 0 } {1}
                    \newLine{ x^2 \p{u+u^3}dx } { -x^3 \p{u^2-1} du } { 2 }
                    \newLine{ -\int \frac{dx}{x} } { \int \frac{u^2-1}{u(1+u^2)} du } { 3 }
                    \\
                        \newLine{ \int \frac{u^2-1}{u(1+u^2)} du } { -\int \frac{ 1 }{ u } du + 2\int \frac{1}{1+u^2} du } { 4.1 }
                        \newLine{ \int \frac{u^2-1}{u(1+u^2)} du } { -ln(u) + 2tan^{-1}u + c} { 4.2 }
                    \\
                    \newLine{ -ln(x) } { -ln(u) + 2tan^{-1}u + c } { 5 }
                    \newLine{ x^{-1} } { Cu^{-1}\e{2tan^{-1}u} } { 6 }
                    \finalAnswer{ x^{-1} } { C\p{\frac{y}{x}}^{-1}\e{2tan^{-1}\p{\frac{y}{x}}} }
                \end{align*}

            \newProblem{o}{ \p{cosxcosy-cotx}dx - \p{senxseny}dy = 0 }
                \begin{align*}
                    \newLine{ \p{cosxcosy-cotx}dx + \p{-senxseny}dy } { 0 } { 1 }
                    \newLine{ \pD{M}{y} = -cosxseny }{ \pD{N}{x} = -cosxseny }{ 2 }
                    \\
                    \newLine{ F(x,y) } { \int \p{cosxcosy-cotx}dx + g(y) } { 3.1 }
                    \newLine{ F(x,y) } { senxcosy - ln(senx) + g(y) } { 3.2 }
                    \\
                    \newLine{ \pD{F}{y} } { N(x,y) } { 4.1 }
                    \newLine{ -senxseny + g'(y) } { senxseny } { 4.2 }
                    \newLine{ g'(y) } { 0 } { 4.3 }
                    \newLine{ g(y) } { C } { 4.4 }
                    \\
                    \newLine{ F(x,y) } { senxcosy - ln(senx) + C } { 5 }
                    \finalAnswer{ senxcosy - ln(senx) }{ C }
                \end{align*}

            \newProblem{p}{ \p{2y^2 + 3x}dx + 2xydy = 0 }
                \begin{align*}
                    \notEqual{ \pD{M}{y} = 4y } { \pD{N}{x} = 2y } { 1 }
                    \\
                    \newLine{ \mu (x) }{ \e{ \int \frac{4y-2y}{2xy} dx} }{ 2.1 }
                    \newLine{ \mu (x) } { \e{ \int \frac{2y}{2xy} dx} } { 2.2 }
                    \newLine{ \mu(x) } { \e{ \int \frac{1}{x} dx} } { 2.3 }
                    \newLine{ \mu(x) } { \e{ ln(x) } } { 2.4 }
                    \\
                        \newLine{ \p{2y^2 x + 3x^2}dx + \p{2x^2y}dy } { 0 } { 3 }
                    \\
                        \newLine{ F(x,y) } { \int \p{2y^2 x + 3x^2}dx + g(y) } { 4.1 }
                        \newLine{ F(x,y) } { y^2 x^2 + x^3 + g(y) } { 4.2 }
                    \\
                        \newLine{ \pD{F}{y} } { N(x,y) } { 5.1 }
                        \newLine{ 2x y^2 + g'(y) } { 2x y^2 } { 5.2 }
                        \newLine{ g'(y) } { 0 } { 5.3 }
                        \newLine{ g(y) } { C } { 5.4 }
                    \\
                    \newLine{ F(x,y) } { y^2 x^2 + x^3 + C } { 6 }
                    \finalAnswer{ y^2 x^2 + x^3 }{ C }
                \end{align*}

        \textbf{2.Resuelva las siguientes aplicaciones:} \\
        \begin{itemize}
            \item Una medicina se inyecta en el torrente sanguíneo de un paciente a un flujo constante de $ r\frac{g}{s} $.
             Al mismo tiempo, esa medicina desaparece con una razón proporcional a la cantidad x(t) presente en cualquier momento t.
             Formule la ecuación diferencial que describa la cantidad x(t) y resuélvala.

             \begin{align*}
                 \newLine{ \frac{dx}{dt} }{ r-kx }{ 1 }
                 \newLine{ \dot{x} + kx }{ r }{ 2 }
                 \\
                 \newLine{ \mu(x) } { \e{\int k dt} } { 3.1 }
                 \newLine{ \mu(x) } { \e{kt} } { 3.2 }
                 \\
                 \newLine{ \dot{x}\e{kt} + kx\e{kt} } { r\e{kt} } { 4 }
                 \newLine{ \frac{d}{dt} \brackets{ x \e{kt} } } { r\e{kt} } { 5 }
                 \newLine{ x \e{kt} } { \int r\e{kt} dt } { 6 }
                 \newLine{ x \e{kt} } { \frac{r}{k} \e{kt} + c} { 7 }
                 \finalAnswer{ x(t) } { \frac{r}{k} + \frac{c}{\e{kt}} }
             \end{align*}

            \item Un tanque tiene 500 gal de agua pura y le entra salmuera con 2 lb de sal por galón a un flujo de 5 gal/min. 
            El tanque está bien mezclado, y sale de él el mismo flujo de solución. 
            Calcule la cantidad A(t) de libras de sal que hay en el tanque en cualquier momento t.

            \begin{align*}
                \newLine{R_in = (2)(5) }{ 10 }{1.1}
                \newLine{ R_out = (\frac{A}{500})(5) }{ \frac{A}{100} }{1.2}
                \newLine{\dot{A}} {10-\frac{A}{100}} {2}
                \newLine{ \dot{A} + \frac{1}{100}A } { 10 } { 3 }
                \\
                \newLine{ \mu (x) }{ \e{\int \frac{1}{100} dt} }{ 4.1 }
                \newLine{ \mu(x) } { \e{\frac{t}{100} } } {4.2}
                \\
                \newLine{ \frac{d}{dt}\brackets{ A \e{\frac{t}{100}} } } { 10\e{\frac{t}{100}} } { 5 }
                \newLine{ A \e{\frac{t}{100}} } { \int 100\e{\frac{t}{100}} dt } { 6 }
                \newLine{ A \e{\frac{t}{100}} } { \e{\frac{t}{100}} + c } { 7 }
                \finalAnswer{A(t)}{ 1+ \frac{c}{\e{\frac{t}{100}}} }
            \end{align*}

            \item $\boldsymbol{q(t) e i(t)}$
            \begin{align*}
                \newLine{ Ri(t) + \frac{1}{c}q} { E(t) } { 1 }
                \newLine{ 200\dot{q} + \frac{1}{10^{-4}}q } { 100 } { 2 }
                \newLine{ \dot{q} + 50q } { \frac{1}{2} } { 3 }
                \\
                \newLine{ \mu (x) }{ \e{\int 50 dt} }{ 4.1 }
                \newLine{ \mu(x) } { \e{50t} } {4.2}
                \\
                \newLine{ \frac{d}{dt}\brackets{ A \e{50t} } } { \frac{1}{2}\e{50t} } { 5 }
                \newLine{ q \e{50t} } { \int \frac{1}{2}\e{50t} dt } { 6 }
                \newLine{ q \e{50t} } { \frac{1}{100}\e{50t} + c } { 7 }
                \finalAnswer{q(t)}{ \frac{1}{100} - \frac{1}{100}\frac{c}{\e{50t}} }
            \end{align*}
        \end{itemize}
    \end{flushleft}

\end{document}