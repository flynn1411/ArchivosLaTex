\documentclass{article}
\usepackage[utf8]{inputenc}
\usepackage[T1]{fontenc}
\usepackage[margin=1in, headsep=0pt]{geometry}
\usepackage{amsmath}
\usepackage{mathtools}
\newcommand{\newLine}[3]{ #1 & = #2 && \text{(#3)}\\ }
\newcommand{\finalAnswer}[2]{ \Aboxed{ #1 & = #2 } }
\newcommand{\p}[1]{ \left( #1 \right) }
\newcommand{\brackets}[1]{ \left[ #1 \right] }
\allowdisplaybreaks
%\usepackage{showframe}
\geometry{letterpaper}
\title{Tarea Unidad 1 MM-411}
\author{Josué Ariel Izaguirre (20171034157)}
\date{Febrero 2020}
\begin{document}
    \maketitle
    \begin{flushleft}
        \textbf{1.Resuelva las siguientes ecuaciones diferenciales:} \break
        
        
        \hspace{10mm} a. $
            \boldsymbol{\left( 1+x^2 \right) \frac{dy}{dx} = \left(1+y \right)^2}
            $  \break

                \begin{align*}
                    ( 1+x^2 ) \frac{dy}{dx} & = (1+y )^2  &&\text{(1)}\\
                    [ ( 1+x^2 ) \frac{dy}{dx} & = (1+y)^2 ] {dx}{(1+x^2)(1+y)^2} &&\text{(2)}\\
                    \frac{dy}{(1+y)^2} & = \frac{dx}{1+x^2} &&\text{(3)}\\
                    \int_{}^{} \frac{dy}{(1+y)^2} & = \int_{}^{} \frac{dx}{1+x^2} &&\text{(4)}\\
                    -\frac{1}{1+y} + c_1 & = \tan ^ {-1} {x} + c_2 &&\text{(5)}\\
                    -\frac{1}{1+y} & = \tan ^ {-1} {x} + c_3, \quad (c_3 = c_2 - c_1) &&\text{(6)} \\
                    -1 - \tan ^ {-1} {x} - c_3 & = y ( \tan ^ {-1} {x} + c_{3} ) &&\text{(7)} \\
                    \Aboxed{y & = -1 - \frac{1}{tan ^{-1}x + c}}
                \end{align*}\\

        \hspace{10mm} b. $
            \boldsymbol{\frac{dy}{dx} = 6e^{2x-y} , y(0) = 0}
            $  \break

            \begin{align*}
                [ \frac{dy}{dx} & = 6\frac{e^{2x}}{e^y} ] (e^y dx) && \text{(1)} \\
                e^y dy & = 6e^{2x} dx && \text{(2)} \\
                \int e^y dy & = \int 6e^{2x} dx && \text{(3)} \\
                e^y & = 3e^{2x} + c && \text{(4)} \\
                y & = Ln(3e^{2x} + c) && \text{(5)} \\
                \\
                y(0) & = 0 && \text{(6.1)} \\
                0 & = Ln(3e^{2(0)} + c) && \text{(6.2)} \\
                1 & = 3 + c && \text{(6.3)} \\
                c & = -2 && \text{(6.4)}\\
                \\
                \Aboxed{y &= Ln(3e^{2x}-2)}
            \end{align*}

        \hspace{10mm} c. $
            \boldsymbol{ xy' + y = 3xy , y(1) = 0}
            $  \break

            \begin{align*}
                xy' & = 3xy - y  && \text{(1)}\\
                y' & = 3y - \frac{y}{x}  && \text{(2)}\\
                \frac{dy}{dx} & = 3y - \frac{y}{x}  && \text{(3)}\\
                [ \frac{dy}{dx} & = 3y - \frac{y}{x} ] (\frac{dx}{y})  && \text{(4)}\\
                \frac{dy}{y} & = (3 - \frac{1}{x})dx  && \text{(5)}\\
                \int \frac{dy}{y} & = \int (3 - \frac{1}{x})dx  && \text{(6)}\\
                ln(y) & = 3x - ln(x) + c  && \text{(7)}\\
                \Aboxed{y & = \frac{e^{3x + c}}{x}}\\
            \end{align*}

        \hspace{10mm} d. $
            \boldsymbol{ (1+x)y' + y = cosx , y(0) = 1}
            $  \break

            \begin{align*}
                [ (1+x)y' + y & = cosx ] (\frac{1}{1+x}) && \text{(1)}\\
                y' + \frac{y}{1+x} & = \frac{cosx}{1+x} && \text{(2)}\\
                \\
                \mu(x) & = e^{\int \frac{1}{1+x} dx} && \text{(3.1)}\\
                \mu(x) & = e^{ln(1+x)} && \text{(3.2)}\\
                \mu(x) & = 1+x && \text{(3.3)}\\
                \\
                [ y' + \frac{y}{1+x} & = \frac{cosx}{1+x} ] (1+x) && \text{(4)}\\
                (1+x)y' + y & = cosx && \text{(5)}\\
                \int \frac{d}{dx} [ (1+x)y ] & = \int cosx dx && \text{(6)}\\
                \newLine{(1+x)y}{senx + c}{7}
                \newLine{y}{\frac{senx + c}{1+x}}{8}
                \\
                \newLine{y(0)}{1}{9.1}
                \newLine{1}{ \frac{sen(0) + c}{1+(0)} }{9.2}
                \newLine{1}{ \frac{c}{1} }{9.3}
                \newLine{c}{1}{9.4}
                \\
                \finalAnswer{y}{\frac{senx + 1}{1+x}}
            \end{align*}

        \hspace{10mm} e. $
            \boldsymbol{ xyy' = y^2 + x\sqrt{4x^2 + y^2} }
            $  \break

            \begin{align*}
                \newLine{[ xy\frac{dy}{dx}} {y^2 x\sqrt{ 4x^2 + y^2} ] (dx) } {1}
                \newLine{xydy} {(y^2 x\sqrt{ 4x^2 + y^2})dx} {2}
                \newLine{xydy - (y^2 x\sqrt{ 4x^2 + y^2})dx } {0} {3}
                \\
                \newLine{u}{\frac{y}{x}}{4.1}
                \newLine{y}{ux}{4.2}
                \newLine{dy}{udx+xdu}{4.3}
                \\
                \newLine{ux^2 (udx + xdu) - (u^2 x^2 + x \sqrt{4x^2 + u^2 x^2})dx}{0}{5}
                \newLine{ x^2(u^2)dx + x^2(-u^2 - \sqrt{4-u^2})dx + x^3du }{0}{6}
                \newLine{ x^2(-\sqrt{4-u^2})dx }{-x^3du}{7}
                \newLine{ x^2(-\sqrt{4-u^2})dx }{-x^3du ] (\frac{1}{(-x^3)(-\sqrt{4-u^2})})}{8}
                \newLine{ \frac{dx}{x} }{ \frac{du}{ \sqrt{4-u^2}} }{9}
                \newLine{\int \frac{dx}{x} }{ \int \frac{du}{ \sqrt{4-u^2}} }{10}
                \newLine{ Ln(x) }{ \int \frac{du}{ \sqrt{4-u^2}} }{11.1}
                \\
                \newLine{2tan\theta = u}{ tan\theta = \frac{u}{2} }{11.1.a}
                \newLine{\sqrt{ 4tan^{2}\theta +4 }}{2sec\theta}{11.1.b}
                \newLine{2sec^{2}\theta d\theta}{du}{11.1.c}
                \newLine{ \int \frac{du}{sqrt{4-u^2}} } { \int \frac{ 2sec^{2}\theta }{ 2sec\theta } } {11.1.d}
                \newLine{ \int \frac{ 2sec^{2}\theta }{ 2sec\theta } } { \int sec\theta d\theta } {11.1.e}
                \newLine{ \int sec\theta d\theta } { ln(tan\theta + sec\theta) + c_1 } {11.1.f}
                \newLine{ \int sec\theta d\theta } { ln( \frac{u}{2} + \frac{ \sqrt{4-u^2} }{2} ) + c_1 } {11.1.g}
                \\
                \newLine{ Ln(x) } { ln( \frac{u + \sqrt{4-u^2} }{2} ) + c_1 } { 12 }
                \newLine{ x } { c \left( u + \sqrt{4-u^2}  \right), c=\frac{e^{c_1}}{2} } { 13 }
                \newLine{ x } { c \left( (\frac{y}{x}) + \sqrt{4-\frac{y^2}{x^2}} \right)  } { 14 }
                \newLine{ x } { c \p{ \frac{y}{x} + \frac{ \sqrt{4x^2 + y^2} }{x} } } { 15 }
                \newLine{ x } { c \p{ \frac{y + \sqrt{4x^2 + y^2} }{x} } } { 16 }
                \finalAnswer{ \frac{ x^2 }{ y + \sqrt{4x^2 + y^2} } }{c}
            \end{align*}
            
    \end{flushleft}

\end{document}