\documentclass{article}
\usepackage[utf8]{inputenc}
\usepackage[T1]{fontenc}
\usepackage[margin=1in, headsep=0pt]{geometry}
\usepackage{amsmath}
\usepackage{mathtools}
%\usepackage{showframe}
\geometry{letterpaper}

\title{Metódos para Resolver Ecuaciones Diferencial Ordinarias de Primer Orden}

\begin{document}

\maketitle

\begin{itemize}
    \item \textbf{Ecuación Diferencial de Variables Separables} \\
    De la forma $\boldsymbol{\frac{dy}{dx} = g(x)h(y)}$. \\
    Resolución
    \begin{align*}
        [ \frac{dy}{dx} & = g(x)h(y) ]*(\frac{dx}{h(y)}) &&\text{(1)}\\
        h(y)dy & = g(x)dx &&\text{(2)}\\
        \int \frac{1}{h(y)} dy & = \int g(x)dx &&\text{(3)}\\
        \Aboxed{ H(y) = G(x) + c } &&\text{ó} && \Aboxed{ H(y) - G(x) = c }
    \end{align*}
    
    \item \textbf{Ecuación Diferencial Exacta} \\
    De la forma $\boldsymbol{M(x,y)dx + N(x,y)dy = 0}$, donde 
    \textbf{$\boldsymbol{\frac{\partial M}{\partial y} = \frac{\partial N}{\partial x}}$} \\
    Resolución \\
    -Usando M(x,y)
    \begin{align*}
        F(x,y) & = \int M(x,y) dx + g(y) &&\text{(1)}\\
        \frac{\partial F}{\partial y}[F(x,y)] & = N(x,y) &&\text{(1.1)}\\
        \frac{\partial F}{\partial y} & = M_y + g'(y) = N(x,y) &&\text{(1.2)}\\
        g'(y) & = N(x,y) - M_y &&\text{(1.3)}\\
        \int g'(y) dy & = \int (N(x,y) - M_y) dy &&\text{(1.4)}\\
        g(y) & = \int (N(x,y) - M_y) dy &&\text{(1.3)}\\
        F(x,y) & = \int M(x,y) dx + \int (N(x,y) - M_y) dy &&\text{(2)}\\
        \Aboxed{ \int M(x,y) dx + \int (N(x,y) - M_y) dy & = c } &&\text{(3)}
    \end{align*}

    -Usando N(x,y)
    \begin{align*}
        F(x,y) & = \int N(x,y) dy + h(x) &&\text{(1)}\\
        \frac{\partial F}{\partial x}[F(x,y)] & = M(x,y) &&\text{(1.1)}\\
        \frac{\partial F}{\partial x} & = N_x + h'(x) = M(x,y) &&\text{(1.2)}\\
        h'(x) & = M(x,y) - N_x &&\text{(1.3)}\\
        \int h'(x) dx & = \int (M(x,y) - N_x) dx &&\text{(1.4)}\\
        h(x) & = \int (M(x,y) - N_x) dx &&\text{(1.3)}\\
        F(x,y) & = \int N(x,y) dy + \int (M(x,y) - N_x) dx &&\text{(2)}\\
        \Aboxed{ \int N(x,y) dy + \int (M(x,y) - N_x) dx & = c } &&\text{(3)}
    \end{align*}

    \item \textbf{Ecuación Diferencial No Exacta}\\
    De la forma $\boldsymbol{M(x,y)dx + N(x,y)dy = 0} $, donde
     \textbf{$\boldsymbol{\frac{\partial M}{\partial y} \neq \frac{\partial N}{\partial x}}$}.\\
    Resolución \\
    Se tiene que encontrar una función $\mu(x)$ ó $\mu(y)$ para multiplicar toda la ecuación 
    con este "factor integrante" y poder tener una ecuación exacta la cual se puede resolver.\\
    \\-Caso $\mu(x)$ \\
    \begin{align*}
        \mu(x) & = e^{\int \frac{M_y - N_x}{N} dx} \\
        [ M(x,y)dx + N(x,y)dy & = 0 ] (\mu(x)) \\
        \Aboxed{M(x,y)\mu(x)dx + N(x,y)\mu(x)dy & = 0}
    \end{align*}
    \\-Caso $\mu(y)$ \\
    \begin{align*}
        \mu(y) & = e^{\int \frac{N_x - M_y}{M} dy} \\
        [ M(x,y)dx + N(x,y)dy & = 0 ] (\mu(y)) \\
        \Aboxed{M(x,y)\mu(y)dx + N(x,y)\mu(y)dy & = 0}
    \end{align*}

    \item \textbf{Ecuación Diferencial Lineal} \\
    De la forma $\boldsymbol{a_1(x)\frac{dy}{dx} + a_0(x)y = g(x)}$.\\
    Esta ecuación se puede escribir en su "forma estándar" al dividir toda la ecuación 
    entre $a_1(x)$. Dejando asi la siguiente ecuación: 
    $\boldsymbol{\frac{dy}{dx} + P(x)y = f(x)}$.\\
    Resolución:\\
    \\i.)Obtner el factor integrante $\mu(x) = e^{\int P(x) dx}$.
    \\ii.)Multiplicar el factor integrante en ambos lados, en el 
    lado izquierdo quedará una derivada del producto del factor 
    integrante y la variable dependiente.\\
    \begin{align*}
        [ \frac{dy}{dx} + P(x)y & = f(x) ] ( \mu(x) ) \\
        \mu(x)\frac{dy}{dx} + P(x)\mu(x)y & = f(x)\mu(x) \\
        \frac{d}{dx} [ \mu(x)*y ] & = f(x)\mu(x)
    \end{align*}
    \\iii.)Integrar a ambos lados y luego despejar para la variable dependiente. 
    Obteniendo así, la solución.\\
    \begin{align*}
        \int \frac{d}{dx} [ \mu(x)*y ] & = \int f(x)\mu(x) dx \\
        \mu(x)y & = \int f(x)\mu(x) dx \\
        \Aboxed{ y & = \frac{1}{\mu(x)}\int f(x)\mu(x) dx } \\
        \Aboxed{ y & = \frac{1}{e^{\int P(x)}}\int f(x)e^{\int P(x)} dx } \\
    \end{align*}

    \item \textbf{Ecuación Diferencial de Coeficientes Homogéneos} \\
    De la forma $\boldsymbol{M(x,y)dx + N(x,y)dy = 0}$ 
    si las funciones $\boldsymbol{M(x,y)}$ y $\boldsymbol{N(x,y)}$  son 
    funciones homogéneas. ( $f(tx,ty) = t^{\alpha}f(x,y)$, función Homogénea 
    de grado $\alpha$ ). \\
    Resolución \\
    \\Se utiliza la sustitución (aunque tambie se puede aplicar $v=\frac{x}{y}$): \\
    \begin{align*}
        u & = \frac{y}{x} && y = ux\\
        && dy & = udx + xdu \\
        M(x,y)dx + N(x,y)dy &= 0 && M(x,ux)dx + N(x,ux)(udx +xdu) = 0
    \end{align*}
    \begin{align*}
        x^{\alpha} M(1,u)dx + x^{\alpha} N(1,u)(udx + xdu) & = 0 \\
        x^{\alpha} M(1,u)dx + ux^{\alpha} N(1,u)dx + ux^{\alpha+1} N(1,u)du & = 0 \\
        x^{\alpha} [ M(1,u) + uN(1,u) ]dx & = -x^{\alpha+1}N(1,u)du \\
        [ x^{\alpha} [ M(1,u) + uN(1,u) ]dx & = -x^{\alpha+1}N(1,u)du ] (-\frac{1}{(x^{\alpha-1})(M(1,u)+N(1,u))}) \\
        -\frac{1}{x} dx & = \frac{N(1,u)}{M(1,u) + uN(1,u)}du \\
        \int -\frac{1}{x} dx & = \int \frac{N(1,u)}{M(1,u) + uN(1,u)}du
    \end{align*}
    \\Deshacer la sustitución y despejar para la variable dependiente de ser posible.\\

    \item \textbf{Ecuación Diferencial de Bernoulli} \\
    De la forma $\boldsymbol{\frac{dy}{dx} + P(x)y = f(x)y^n}$, donde 
    $n \neq 0, n \neq 1$. \\
    Resolución \\
    \\ Utilizar la sustitución: \\
    \begin{align*}
        u & = y^{1-n} && y = u^{\frac{1}{1-n}} \\
        du & = (1-n)y^{-n} && dy = (\frac{1}{1-n})(u^{\frac{1}{1-n}-1})du
    \end{align*}
    \\Luego quedará una ecuación diferencial de las anteriormente vistas para resolver con 
    su resolución acorde.

    \item \textbf{Ecuación Diferencial con Coeficientes Lineales} \\
    De la forma $\boldsymbol{ (ax + by + c)dx + (\alpha x + \beta y + \gamma )dy = 0 }$ ó tambien 
    $\boldsymbol{ \frac{dy}{dx} = \frac{ax + by + c}{\alpha x + \beta y + \gamma } }$. \\
    Resolución \\
    \\ Resolver el sistema de ecuaciones: \\
    $ \begin{cases}
        ax + by + c \\ 
        \alpha x + \beta y + \gamma  
       \end{cases} 
    $ \\
    Con la solución $ (h,k) $. Con la solución se procede a realizar la sustitución: \\
    \begin{align*}
        x & = u + h && y  = v + k \\
        dx & = du && dy  = dv
    \end{align*}\\
    La sustitución anterior dejará la ecuación en una forma $\boldsymbol{ (au + bv)du + (\alpha u + \beta v)dv = 0 }$ ó tambien 
    $\boldsymbol{ \frac{dv}{du} = \frac{au + bv }{\alpha u + \beta v} }$ la cual se puede resolver utilizando métodos anteriormente vistos
    (principalmente homogéneas, exactas y/o no exactas).
\end{itemize}

\end{document}
